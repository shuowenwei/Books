%% LyX 2.0.0 created this file.  For more info, see http://www.lyx.org/.
%% Do not edit unless you really know what you are doing.
\documentclass[english]{article}
\usepackage[T1]{fontenc}
\usepackage[utf8]{luainputenc}
\usepackage{color}
\usepackage{amssymb}
\usepackage{babel}
\begin{document}

\title{Midterm Exam}


\author{Shuowen Wei}
\maketitle
\begin{description}
\item [{Problem}] 1
\end{description}
Proof:

By the given conditions, we can easily get: $Ir+Ax=b$ and $A^{T}r=0$.
Then left multiply $A^{T}$ to the first equation, we have,

\[
A^{T}Ir+A^{T}Ax=0+A^{T}Ax=A^{T}b
\]


thus $A^{T}Ax=A^{T}b$, since $A$ is full rank, then by \textbf{Theorem
11.2}, we know that the solution $x$ of $A^{T}Ax=A^{T}b$ minimize
the $||Ax-b||_{2}$. 
\begin{description}
\item [{Problem}] 2
\end{description}
Solution:

Since $A$ is a real matrix with reduced SVD $A=\hat{U}\hat{\Sigma}V^{T}$,
so $\hat{U}$ is m-by-n real unitary matrix and $\hat{\Sigma,\, V}$
are both n-by-n real matrix with is $\Sigma$ diagonal and $V$ is
unitary. 

Thus, we have:

1.
\[
(A^{T}A)^{-1}A^{T}=(V\hat{\Sigma}\hat{U}^{T}\hat{U}\hat{\Sigma}V^{T})^{-1}(\hat{U}\hat{\Sigma}V^{T})^{T}=(V\hat{\Sigma}^{-2}V^{T})(V\hat{\Sigma}\hat{U}^{T})=V\hat{\Sigma}^{-1}\hat{U}^{T}(=A^{-1})
\]


2.

\[
A(A^{T}A)^{-1}=(\hat{U}\hat{\Sigma}V^{T})(V\hat{\Sigma}\hat{U}^{T}\hat{U}\hat{\Sigma}V^{T})^{-1}=(\hat{U}\hat{\Sigma}V^{T})(V\hat{\Sigma}^{-2}V^{T})=\hat{U}\hat{\Sigma}^{-1}V^{T}(=(A^{T})^{-1})
\]

\begin{description}
\item [{Problem}] 3
\end{description}
Proof:

For an arbitrary real matrix $A\in\mathbb{C}^{m\times n}$, Moore-Penrose
pseudoinverse of $A$ ($A$ must be full rank), denoted by $A^{+}$,
is defined as: 

\[
A^{+}=(A^{T}A)^{-1}A^{T}\in\in\mathbb{C}^{n\times m}
\]


Since real matrix $A$ has a reduced SVD $A=\hat{U}\hat{\Sigma}V^{T}$,
then by using the results of problem 2, we verify:

\[
AA^{+}A=A((A^{T}A)^{-1}A^{T})A=(\hat{U}\hat{\Sigma}V^{T})(V\hat{\Sigma}^{-1}\hat{U}^{T})(\hat{U}\hat{\Sigma}V^{T})=\hat{U}\hat{\Sigma}V^{T}=A
\]


and

\[
A^{+}A=((A^{T}A)^{-1}A^{T})A=(V\hat{\Sigma}^{-1}\hat{U}^{T})(\hat{U}\hat{\Sigma}V^{T})=I=I^{T}=(A^{+}A)^{T}
\]


Thus the Moore-Penrose pseudoinverse of $A$ satisfies those two identities. 
\begin{description}
\item [{Problem}] 4
\end{description}
Proof:

It is very similar to \textbf{Prop 11.2}, let $Az=0$, then if $x$
minimizes $||Ax-b||_{2}$, so does $x+z$. And $z$ is any element
of $null(A)$, since $A\in\mathbb{C}^{m\times n}$ is full rank, then
$rank(A)=m$ since $m<n$. So $dim(null(A))=n-m$, thus the solution
$x+z$ is an (n-m)-dimentional set. 

Since $A\in\mathbb{C}^{m\times n}(m<n)$, so $A^{*}A\mathbb{\in C}^{n\times n}$
is not nonsingular any more because its rank is $m$ and $m<n$. But
$AA^{*}\mathbb{\in C}^{m\times m}$ is nonsingular and invertible.
So it is nature to think of another way to set $x=A^{*}y$ such that
$AA^{*}y=b$. Once we compute $y$, we get $x$. And it is easy to
find that $y=(AA^{*})^{-1}b$, thus one solution of the underdetermined
least square problem is:

\[
x=A^{*}y=A^{*}(AA^{*})^{-1}b
\]


First of all, we will prove that this solution is exactly the unique
minimum norm solution.

Suppose that for an arbitrary solution $X$ of this problem, i.e.,
$AX=b$, then we have $A(X-x)=0$. Thus

\[
(X-x)^{*}x=(X-x)^{*}A^{*}(AA^{*})^{-1}b=(A(X-x))^{*}(AA^{*})^{-1}b=0
\]


then $X-x$ and $x$ are mutually orthogonal, so we have $||X||^{2}=||X-x+x||^{2}=||X-x||^{2}+||x||^{2}\geqslant||x||^{2}$,
since $X$ is the arbitrary solution of this problem, then $x$ is
the unique minimum norm solution. 

Secondly, we will give out different algorithms to compute this mim-norm
solution:

\textbf{Appropriately modified normal equations:}

1. Form $AA^{*}$

2. Compute the Cholesky factorization $AA^{*}=R^{*}R$

3. Solve the lower-triangular system $R^{*}w=b$ for $w$

4. Solve the upper-triangular system $Rz=w$ for $z$

5. Set $x=A^{*}z$ 

\textbf{QR decomposition:}

1. Compute the reduced QR factorization $A^{*}=\hat{Q}\hat{R}$, where
$\hat{Q}\in\mathbb{C}^{n\times m},\, and\,\hat{R}\in\mathbb{C}^{m\times m}$,
$\hat{Q}$ is unitary and $\hat{R}$ is nonsingular, upper triangular

2. Solve the lower-triangular system $\hat{R}^{*}z=b$ for $z$

3. Set $x=\hat{Q}z$

\textbf{SVD:}

1. Compute the reduce SVD $A^{*}=\hat{U}\hat{\Sigma}V^{*}$

2. Compute the vector $\hat{V}^{*}b$

3. Solve and $\hat{\Sigma}w=\hat{V}^{*}b$ for $w$

4. Set $x=\hat{U}w$
\begin{description}
\item [{Problem}] 11.1
\end{description}
Proof:

Since $A_{1}$is nonsingular matrix of dimension $n\times n$, then
$A$ is full rank, so by the definition of pseudoinverse of $A$,
denoted by $A^{+}$, $A^{+}=(A^{*}A)^{-1}A^{*}$. And $A$ has reduced
SVD $A=\hat{U}\hat{\Sigma}V^{*}$, then $A^{*}A=V\Sigma^{2}V^{*}$,
so we have

\[
||A^{+}||_{2}=||(A^{*}A)^{-1}A^{*}||_{2}=||V\Sigma^{-2}V^{*}V\Sigma U^{*}||_{2}=||V\Sigma^{-2}V^{*}V\Sigma U^{*}||_{2}=||\Sigma^{-1}||_{2}=\frac{1}{\sigma_{n}}
\]


where $\sigma_{n}$ is the smallest singular value, because $\sigma_{1},\,\sigma_{2},\,...\sigma_{n}$
are in decreasing order. 

Since $V\Sigma^{2}V^{*}=A^{*}A=\left[\begin{array}{cc}
A_{1}^{*} & A_{2}^{*}\end{array}\right]\left[\begin{array}{c}
A_{1}\\
A_{2}
\end{array}\right]=A_{1}^{*}A_{1}+A_{2}^{*}A_{2}$, then 
\[
\Sigma^{2}=V^{*}(A_{1}^{*}A_{1}+A_{2}^{*}A_{2})V=V^{*}A_{1}^{*}A_{1}V+V^{*}A_{2}^{*}A_{2}V=\Sigma_{1}^{2}+\Sigma_{2}^{2}
\]


where $\Sigma_{1}$ and $\Sigma_{2}$ are the singular value matrix
from the SVD of $A_{1}$and $A_{2}$.

Because $A_{1}$ is full rank, thus every element in $\Sigma_{1}^{2}$
is positive, denote by $\alpha_{i}^{2},\, i=1,2...n$. And denote
the element in $\Sigma_{2}^{2}$ as $\beta_{i}^{2},\, i=1,2...n$.
Both $\alpha_{i}^{2}$ and $\beta_{i}^{2}$ are in decreasing order.
So

\[
\sigma_{i}^{2}=\alpha_{i}^{2}+\beta_{i}^{2},\, i=1,2...n
\]


Since $A_{2}$ is arbitrary, then $\beta_{i}^{2}\geq0$, (i.e some
$\beta_{i}^{2}$ may equals to zero ), thus $\sigma_{i}^{2}\geq\alpha_{i}^{2},\, i=1,2...n$,
let $i=n$, then $\frac{1}{\sigma_{n}}\leq\frac{1}{\alpha_{n}}=||A_{1}^{-1}||_{2}$,
hence

\[
||A^{+}||_{2}\leq||A_{1}^{-1}||_{2}
\]

\begin{description}
\item [{Problem}] 11.3
\end{description}
Please see the m-file: \textbf{midterm.m}

\textbf{Just copy the whole code and run it, we will get as follows:}

the solution 1 is very good and acceptable 

the solution 2 is very good and acceptable 

the solution 3 is very good and acceptable 

the solution 4 is very good and acceptable 

the solution 5 is very good and acceptable 

the solution 6 is very good and acceptable

===== Please input \textquotedbl{}result\textquotedbl{} directly in
the Command Window to get all the solutions if you like ===== 

Below are the six lists of the twelve coefficients: 

coeff = Columns 1 through 3 

\textcolor{red}{999.999961392580e-003} 1.00000000166766e+000 1.00000000099660e+000 

\textcolor{red}{934.609598937149e-003} 934.609643543441e-003 934.609642735242e-003 

746.990399160323e-003 746.990362478313e-003 746.990363182523e-003 

461.679130551934e-003 461.679153132704e-003 461.679152633007e-003 

115.989201689278e-003 115.989207162564e-003 115.989207125290e-003 

-244.869859492343e-003 -244.869886874760e-003 -244.869886336078e-003 

-573.704742484316e-003 -573.704721069467e-003 -573.704721555216e-003 

-827.510048967594e-003 -827.510044386540e-003 -827.510044432986e-003 

\textcolor{red}{-973.092981738117e-003} -973.093006396147e-003 -973.093005888365e-003 

-991.414199474889e-003 -991.414170358503e-003 -991.414170994523e-003 

-880.077444206421e-003 -880.077476583473e-003 -880.077475870130e-003

\textcolor{red}{-653.643593875715e-003} -653.643620146207e-003 -653.643619561760e-003 

Columns 4 through 6 

1.00000000099661e+000 1.00000000099661e+000 1.00000000099661e+000 

934.609642735245e-003 934.609642735245e-003 934.609642735244e-003 

746.990363182526e-003 746.990363182526e-003 746.990363182525e-003 

461.679152633008e-003 461.679152633008e-003 461.679152633008e-003 

115.989207125290e-003 115.989207125291e-003 115.989207125290e-003 

-244.869886336077e-003 -244.869886336077e-003 -244.869886336078e-003 

-573.704721555216e-003 -573.704721555216e-003 -573.704721555216e-003

-827.510044432986e-003 -827.510044432985e-003 -827.510044432987e-003 

-973.093005888366e-003 -973.093005888364e-003 -973.093005888366e-003 

-991.414170994525e-003 -991.414170994522e-003 -991.414170994525e-003 

-880.077475870132e-003 -880.077475870130e-003 -880.077475870132e-003

-653.643619561763e-003 -653.643619561760e-003 -653.643619561764e-003

The mutual differences between the observations are very small, such
that we can even ignor the differences and think they are the same.
And the normal equations, precisely speaking, exihits a little instability,
but its solution can be acceptable. 
\end{document}
